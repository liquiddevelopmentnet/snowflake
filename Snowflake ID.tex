\documentclass{article}

\usepackage[english]{babel}
\usepackage[a4paper,top=2cm,bottom=2cm,left=3cm,right=3cm,marginparwidth=1.75cm
]{geometry}
\usepackage{amsmath}
\usepackage[utf8]{inputenc}
\usepackage{graphicx}
\usepackage[colorlinks=true, allcolors=blue]{hyperref}
\usepackage{xcolor}
\usepackage{titling}
\usepackage{float}
\usepackage{tikz}
\usepackage[RPvoltages]{circuitikz}
\usepackage{fancyvrb}
\usepackage{newverbs}

\addto\captionsenglish{
	\renewcommand{\contentsname}%
	{Table of Contents}%
}

\newcommand{\code}[1]{\colorbox{cverbbg}{\texttt{#1}}}
\newcommand{\hn}[0]{\hfill \\}

\definecolor{cverbbg}{gray}{0.93}

\newenvironment{lcverbatim}
{\SaveVerbatim{cverb}}
{\endSaveVerbatim{}
	\flushleft\fboxrule=0pt\fboxsep=.5em
	\colorbox{cverbbg}{%

		\makebox[\dimexpr\linewidth-2\fboxsep][l]{\BUseVerbatim{cverb}}%
	}
	\endflushleft{}
}

\title{\textbf{Snowflake Id}}
\author{Finn Behrend}

\begin{document}
\begin{titlingpage}
	\maketitle
	\begin{center}
		\vspace{50px}
		Complete sheet and specification about Twitter's Snowflake Id
		concept
		\vfill
		\footnotesize{Copyright \copyright{} 2023 --- Finn Behrend
			(https://github.com/liquiddevelopmentnet)}
	\end{center}
\end{titlingpage}

\tableofcontents

\pagebreak

\section{Introduction}
The `Snowflake' concept was originally invented by
\href{https://twitter.com}{Twitter} and is designed to generate short and
guaranteed unique Id\footnote{Short for \textbf{Identifier},  is a name
	that
	identifies (that is, labels the identity of) either a unique object or
	a
	unique
	class of objects, where the `object' or class may be an idea, physical
	countable object (or class thereof), or physical non-countable
	substance
	(or
	class thereof).} numbers at high scale.

\subsection{Motivation}
Twitter invented the Snowflake concept because they were moving away from
\href{https://en.wikipedia.org/wiki/MySQL}{MySQL}\footnote{MySQL is an
	open-source relational database management system. Its name is a
	combination of
	`My', the name of co-founder Michael Widenius's daughter My, and `SQL',
	the
	acronym for Structured Query Language.} and moved to
\href{https://en.wikipedia.org/wiki/Cassandra}{Cassandra}\footnote{Apache
	Cassandra is
	a free and open-source, distributed, wide-column store, NoSQL database
	management system designed to handle large amounts of data across many
	commodity servers, providing high availability with no single point of
	failure.} because of scaling
reasons, so they needed a new way to generate unique Id's based on the fact
that Cassandra does and should not provide a sequential Id generation facility.

\section{Capacity}
A Snowflake Infrastructure is designed to withstand the following specifications with ease:
\begin{itemize}
	\item 10k Id generations per second per process
	\item Response rate of 2ms
\end{itemize}
In large data centers, machines generating Id's should not have to coordinate
with each other. That's what makes Snowflakes Concept so brilliant as computers
do not have to communicate with each other during or after the Id creation
process.

\section{Scheme}
Example of a Snowflake Id from
\href{https://discord.com/}{Discord}\footnote{Discord is a VoIP and instant
	messaging social platform. Users have the ability to communicate with
	voice
	calls, video calls, text messaging, media and files in private chats or
	as
	part
	of communities called `servers'.} we'll use throughout the sheet:
\code{555780371086704681} \\
\hn{}
A Snowflake Id consists of a 64-bit number (e.g.\ a \code{uint64}), depending
on
implementation they mostly are handled as strings to prevent integer overflows
in some languages. \\
\hn{}
Being a 64-bit integer makes the Snowflake Id's practical maximum
$2^{64} - 1$ (or $($FFFFFFFFFFFFFFFF$)_{16}$),
(see \nameref{snowflake_id_constraints}).

\section{Binary Breakdown}
A Snowflake Id is constructed by chaining binary data and then converting to
decimal. \\

\begin{figure}[H]
	\large{\texttt{\color{cyan}000001111011011010000111010011011000011111
			\color{red}00001 \color{green}00000
			\color{gray}000000101001}}
	\\
	\caption{Binary Breakdown of the example
		Id}\label{fig:breakdown_example}
\end{figure}

\begin{table}[h]
	\begin{tabular}{l|c|l}
		Field                & Bits           & Description
		\\ \hline
		Timestamp            & \color{cyan}42 & Millisenconds since the
		standard UNIX Epoch
		\\ \hline
		Internal worker Id   & \color{red}5   & Usually incremented Id
		of the
		worker
		\\ \hline
		Internal process Id  & \color{green}5 & Usually incremented Id
		of the
		process in
		the worker
		\\ \hline
		Increment / Sequence & \color{gray}12 & For every Id generated
		on
		process, this number is incremented
	\end{tabular}
\end{table}

\pagebreak

\section{Retrieving data from a Snowflake Id}

\subsection{Timestamp Field}
The timestamp field is the first field and has \code{42} bits in total. It
contains the UNIX Timestamp\footnote{Unix time is a date and time
	representation widely used in computing. It measures time by the number
	of
	seconds or milliseconds that have elapsed since \texttt{00:00:00 UTC on
		1
		January 1970}, the beginning of the Unix epoch, less
	adjustments made
	due to
	leap seconds.} when the Snowflake Id was generated. Implementations
like to
use
custom epochs to lower the timestamp at the beginning of use, and postpone the
`Year-2038' problem. Usually the first day of January in a Year is taken, (see \nameref{year_2038_problem}).

\begin{figure}[H]
	\begin{lcverbatim}
	long snowflake = 555780371086704681;
	long customEpoch = 1420070400000;
	// As corresponding example for the Snowflake Id,
	// Discord's custom Epoch is used

	long timestamp = (snowflake >> 22) + customEpoch;
	\end{lcverbatim}
	\caption{Example of timestamp retrieval in C with custom
		epoch}\label{fig:r_timestamp_1}
\end{figure}

\begin{figure}[H]
	\begin{lcverbatim}
	long snowflake = 555780371086704681;
	long timestamp = (snowflake >> 22);
	\end{lcverbatim}
	\caption{Example of timestamp retrieval in C without custom
		epoch}\label{fig:r_timestamp_2}
\end{figure}

\subsection{Internal worker-/process Id Fields}
The `Internal worker Id' and `Internal process Id' fields are the second and
third fields and both hold 5 bits of data. \\
\hn{}
These fields hold simply what their names tell us, the former field holds the
unique Id of the worker machine that created the Snowflake, and the second
field holds the unique process Id living in that worker machine, both usually
incremented. \\

\begin{figure}[H]
	\begin{lcverbatim}
	long snowflake = 555780371086704681;
	int workerId = (snowflake & 0x3E0000) >> 17;
	\end{lcverbatim}
	\caption{Example of worker Id retrieval in C}\label{fig:r_worker}
\end{figure}

\begin{figure}[H]
	\begin{lcverbatim}
	long snowflake = 555780371086704681;
	int processId = (snowflake & 0x1F000) >> 12;
	\end{lcverbatim}
	\caption{Example of process Id retrieval in C}\label{fig:r_process}
\end{figure}

\pagebreak

\subsection{Increment Field}
The increment field, with consisting of 12 bits is the last chained data in the
Snowflake Id.\@ It is incremented for every Id that is generated on that
process.

\begin{figure}[H]
	\begin{lcverbatim}
	long snowflake = 555780371086704681;
	int increment = snowflake & 0xFFF;
	\end{lcverbatim}
	\caption{Example of increment retrieval in C}\label{fig:r_inc}
\end{figure}

\subsection{Complete Implementation Example}
This is what a proper code retrieving all fields from a Snowflake Id may look
like: \\

\begin{figure}[H]
	\begin{lcverbatim}
	long snowflake = 555780371086704681;
	long customEpoch = 1420070400000;

	long timestamp = (snowflake >> 22) + customEpoch;
	int workerId = (snowflake & 0x3E0000) >> 17;
	int processId = (snowflake & 0x1F000) >> 12;
	int increment = snowflake & 0xFFF;
	\end{lcverbatim}
	\caption{Example of proper retrieval implementation with custom epoch
		in
		C}\label{fig:r_complete_1}
\end{figure}

\begin{figure}[H]
	\begin{lcverbatim}
	long snowflake = 555780371086704681;

	long timestamp = (snowflake >> 22);
	int workerId = (snowflake & 0x3E0000) >> 17;
	int processId = (snowflake & 0x1F000) >> 12;
	int increment = snowflake & 0xFFF;
	\end{lcverbatim}
	\caption{Example of proper retrieval implementation without custom
		epoch in
		C}\label{fig:r_complete2}
\end{figure}

\pagebreak

\section{Generating Snowflake Id's}

\begin{figure}[H]
	\centering
	\resizebox{1\textwidth}{!}{%
		\begin{circuitikz}
			\tikzstyle{every node}=[font=\footnotesize]
			\draw [ -Stealth] (9.75,10.75) -- (12.25,10.75);
			\draw  (8.5,10.75) circle (1cm) node {\Large User} ;
			\draw  (17.5,12.5) rectangle  node {\normalsize Machine
					1}
			(20.75,11.5);
			\draw  (17.5,11.25) rectangle  node {\normalsize
					Machine 2}
			(20.75,10.25);
			\draw  (17.5,10) rectangle	node {\normalsize
					Machine 3} (20.75,9);
			\draw [, dashed] (12.5,10.25) rectangle  node
				{\normalsize Relay}
			(14.75,11.25);
			\draw (16.75,10.75) to[short, -*] (16.75,10.75);
			\draw [](15,10.75) to[short] (16.75,10.75);
			\draw [](16.75,10.75) to[short] (16.75,12);
			\draw [](16.75,10.75) to[short] (16.75,9.5);
			\draw [ -Stealth] (16.75,12) -- (17.25,12);
			\draw [ -Stealth] (16.75,10.75) -- (17.25,10.75);
			\draw [ -Stealth] (16.75,9.5) -- (17.25,9.5);
		\end{circuitikz}
	}%

	\caption{Example Infrastructure of a Id generating
		network}\label{fig:ex_infrastructure}
\end{figure}

Assuming we have a infrastructure as shown above, and a user is requesting a
registration that gets sent to a relay, which then relays the request of one of
our 3 machines (which generates a Id for that user), and let our Id generating
function look like this:

\begin{figure}[H]
	\begin{lcverbatim}
	static long increment_mask = -1 ^ (-1 << 12);

	unsigned long snowflake(int workerId, int processId)
	{
		long timestamp = unix_time();
	
		static int increment = -1;
		increment++;
		increment = increment & increment_mask;
		// Mask to 12 bits to prevent overflow
	
		return (timestamp << 22)
			| (workerId << 17)
			| (processId << 12)
			| increment;
	}
	\end{lcverbatim}
	\caption{Basic Example of a snowflake generating function in
		C}\label{fig:ex_gen_function}
\end{figure}
\hn{}
As you can see, we have two arguments that can be passed into that function, a
\code{workerId} variable and a \code{processId} variable. Every worker machine
should have a different worker Id, while every process or thread of that worker
should have a different process Id (both usually incremented).\\
\hn{}
First, we are getting the current UNIX Timestamp. It is highly recommended to
use NTP\footnote{The Network Time Protocol (short \textbf{NTP}) is a networking
	protocol for clock synchronization between computer systems over
	packet-switched, variable-latency data networks.} to keep your system
clock
accurate. \\
\hn{}
Now we are defining a static integer to make a consistent incremental for each
call of the function, and increase it by 1 after initialisation. \\
\hn{}
At last, we are shifting the data into the right positions and return the final
Snowflake Id.\@

\pagebreak

\subsection{Practical Test}
I've let the first machine generate three snowflakes, two immediately after
each other and one delayed 2 seconds after the first two Id's.

\begin{figure}[H]
	\begin{lcverbatim}
	7046923580470329344
	7046923580474523649
	7046923588863131650
	\end{lcverbatim}
	\caption{The three Snowflake Id's that were
		generated}\label{fig:output_1}
\end{figure}
\hn{}
Let's break them down. \\

\begin{figure}[H]
	\large{\texttt{\color{cyan}11000011100101110110011101010010000010101
			\color{red}00001 \color{green}00000
			\color{gray}000000000000}}
	\\
	\caption{Binary Breakdown of the first Id}\label{fig:breakdown_1}
\end{figure}

\begin{figure}[H]
	\large{\texttt{\color{cyan}11000011100101110110011101010010000010110
			\color{red}00001 \color{green}00000
			\color{gray}000000000001}}
	\\
	\caption{Binary Breakdown of the second Id}\label{fig:breakdown_2}
\end{figure}

\begin{figure}[H]
	\large{\texttt{\color{cyan}11000011100101110110011101010101111100110
			\color{red}00001 \color{green}00000
			\color{gray}000000000010}}
	\\
	\caption{Binary Breakdown of the third Id}\label{fig:breakdown_3}
\end{figure}

\hn{}
As you can see, all Id's were generated on worker one and process zero (red and
green fields). \\
\hn{}
Also, the increment field properly incremented from one, to two, to three.
\\
\hn{}
Now, let's look at the timestamps (non-custom epoch):

\begin{figure}[H]
	\begin{lcverbatim}
	1680117507093
	1680117507094
	1680117509094
	\end{lcverbatim}
	\caption{The timestamps of the three Snowflake Id's that were
		generated}\label{fig:breakdown_timestamps}
\end{figure}
As you can see, on the first two generations the timestamp only changed by
\code{1 ms} which is pure machine latency.
But looking at the difference from the second and third timestamp we see a
\code{2000 ms} difference, which is exactly what we wanted! \\
\hn{}
For a custom epoch, just subtract your epoch from the UNIX epoch and then
compile to a Snowflake.

\break{}

\section{The 2038 Problem}
\label{year_2038_problem}
The 2038 problem, also known as the Unix Y2K bug, refers to a limitation of the
Unix timestamp, which is used to represent time as a 32-bit signed integer, and
is set to overflow on \code{January 19, 2038}. This means that any software
relying on the Unix timestamp to represent dates beyond that point will
encounter errors or fail completely. \\
\hn{}
As we already know, a Snowflake is a 64-bit integer and a part of it consists
of a constantly growing UNIX Timestamp. Since the Snowflake is in fact able to
hold a 42-bits instead of just 32, the problem gets put off to \code{May 15,
	2109} but, is still present. \\
\hn{}
One approach, is to use custom epochs which postpone the problem for a time. A
good choice for a custom epoch is a time before you started utilising Snowflake
Id's, since Snowflake Id's usually never require to have a timestamp that goes
back before the implementation time. \\
Of course, this is not the case if you already have and plan to migrate from an
old Id System, and want to convert Id's. \\
\hn{}
After all, those approaches only delay the problem, so its always good to have
it back in mind.

\section{Constraints}

\subsection{Timestamp Field}
The timestamp is a 42-bit integer, which means that it can hold a value from
\code{0} to \code{4398046511103} (or $2^{42} - 1$).

\subsection{Worker Id and Process Id Field}
The worker Id and process Id are both 5-bit integers, which means that they
can hold a value from \code{0} to \code{31}. \\
\hn{}
The worker Id is used to identify the machine that generated the Id, while the
process Id is used to identify the process or thread that generated the Id.\@

\subsection{Increment Field}
The increment is a 12-bit integer, which means that it can hold a value from
\code{0} to \code{4095}.

\subsection{Snowflake Id}
\label{snowflake_id_constraints}
The Snowflake Id is a 64-bit integer, which means that it can hold a value from
\code{0} to \code{18446744073709551615} (or $2^{64} - 1$).

\section{Sources}
\begin{small}
	\begin{itemize}
		\item

		      \url{https://github.com/twitter-archive/snowflake/tree/snowflake-2010#readme}
		\item \url{https://en.wikipedia.org/wiki/Identifier}
		\item \url{https://en.wikipedia.org/wiki/MySQL}
		\item \url{https://en.wikipedia.org/wiki/Apache_Cassandra}
		\item \url{https://en.wikipedia.org/wiki/Discord}
		\item \url{https://en.wikipedia.org/wiki/Network_Time_Protocol}
		\item \url{https://en.wikipedia.org/wiki/Unix_time}
		\item \url{https://en.wikipedia.org/wiki/2038_problem}
	\end{itemize}
\end{small}

\end{document}